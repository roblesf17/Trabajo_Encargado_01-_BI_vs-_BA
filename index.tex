<<<<<<< HEAD

\documentclass[preprint,12pt]{elsarticle}

\usepackage[spanish]{babel}
\usepackage{amssymb}
\usepackage{graphicx}
\usepackage{lineno}
\usepackage[utf8]{inputenc}
\usepackage{url}
\usepackage{natbib}

\begin{document}
	
	\begin{frontmatter}

		\title{\huge  COMPARATIVA ENTRE INTELIGENCIA DE NEGOCIO (BI) Y ANALITICA DE NEGOCIO(BA) }
		\author{Robles Flores, Anthony Richard	                (2016056192)}
		\author{Estrella Palacios, Katherine Lizbeth			(2015050948)}
		\author{Sosa Bedoya, Sharon					(2016054460)}
		\author{Torres Beltran , Joihanna				(2015053235)}
		\address{Tacna, Perú}
		


%%INICIO abstract
\begin{abstract}
Aqui ira el abstract
\\
Aqui ira el abstract
\end{abstract}
%%FIN abstract


\end{frontmatter}

%%INICIO Resumen
\section{Resumen}
aqui va el resumen
\\
\\
aqui va el resumen
%%FIN Resumen


%%INICIO Introducción
\section{Introducción}
El contexto de la sociedad de la información a propiciado tener la necesidad de mejores, más rápido y eficientes 
métodos para extraer y transformar los datos de una organización en información y distribuirla a lo largo de la 
cadena de valor.
\\
\\
En este articulo podremos apreciar los conceptos base acerca de la inteligencia de negocio (Business Intelligence)
 donde responde a esta como una necesidad y podemos entender en una primera aproximación que es una 
evolución de los sistemas de soporte a la decisiones (DSS).
%%FIN Introducción


%%INICIO Marco Teórico
\section{Marco Teórico}

%%----------------------------------------------------------------------------------------------------------------------------------------------------------
	\subsection{Inteligencia de Negocios (BI)}
	Hay que tomar en cuenta que este concepto Business Intelligence es un tema que viene desde octubre de 1958 
por Hans Peter Luhn (Investigador de IBM). Este concepto ha evolucionado aunando diferentes tecnologías, metodologías
 y términos.\cite{referenciarobles1}
\\
\\
Business Intelligence  es un conjunto de metodologías, aplicaciones, prácticas y capacidades enfocadas a la creación
 y administración de información que permite tomar las mejores decisiones a los usuarios en una organización.
\\
\\
Algunas de las tecnologías que forman parte de Business Intelligence son:
\\
\\
	\begin{itemize}
	\item Data wareHouse 
	\item Reporting 
	\item Análisis OLAP
	\item Análisis Visual
	\item Análisis Predictivo 
	\item Cuadro de mando
	\item Cuadro de mando integral
	\item Minera de datos 
	\item Gestión de rendimiento
	\item Reglas de negocio
	\item Dashboards 
	\item Integracion de Datos
	\end{itemize}



%%****
	\subsection{Beneficios de un sistema de Inteligencia de Negocio (BI)}
	La implantación de estos sistemas de información proporciona diversos beneficios entre los que podemos destacar:
\\
\\
	\begin{itemize}
	\item Crear un circulo virtuoso de la información (Donde los datos se transforman en información que permitirá 
		generar conocimiento para una toma de decisiones que se traducirán en mejores resultados y que 
		generarán nuevos datos).
	\item Permite una visión única, conformada, histórica, persistente y de calidad de toda la información. 
	\item Crear y manejar métricas, indicadores claves de rendimiento, e indicadores claves de meta fundamentales 
		para la empresa.
	\item Aportar información actualizada tanto a nivel agregado como en detalle.
	\item Reducir el diferencial de orientación de negocio entre el departamento de TI y la organización. 
	\item Mejorar comprensión y documentación de los sistemas de información en el contexto de una organización.
	\item Mejorar la competitividad de la organización como resultado de ser capaces de diferenciar lo relevante sobre 
		lo superfluo, acceder más rápido a la información, tener mayor agilidad en la toma de decisiones.
	\end{itemize}
%%****
	\subsection{La necesidad de la Inteligencia de Negocio}
	Existen situaciones en las que la implantación de un sistema Business Intelligence  resulta adecuada. Podríamos destacar 
	entre las que existen son:
\\
\\
	\begin{itemize}
	\item La toma de decisiones se realiza de forma intuitiva en la organización resultados y que generarán nuevos datos).
	\item Identificación de problema de calidad de información
	\item Uso de Excel como repositorios de información corporativos o de usuario o conocido como el Excel Caos.
	\item Necesidad de cruzar información de forma ágil entre departamentos.
	\item Evitar sirios de información.
	\item Las campañas de marketing no son efectivas por la información base usada. 
	\item Existe demasiada información en la organización para ser analizada de la forma habitual (se alcanzó la masa crítica de datos).
	\end{itemize}
%%****
	\subsection{Estrategia de Business Intelligence}
	Desplegar un proyecto de inteligencia de negocio en una organización no es un proceso sencillo. Las buenas practicas dicen 
	que para llegar a un buen puerto, es necesario tener una estrategia de inteligencia de negocio que coordine de forma efectiva 
	las tecnologías, el uso, los procesos de madurez.\cite{referenciarobles2}
%%****
	\subsection{Ausencia de una estrategia de BI}
	Es posible detectar que no existe estrategia definida a través de los siguientes puntos:
\\
\\
	\begin{itemize}
	\item Los usuarios identifican al departamento de TI como el origen de problemas de inteligencia de negocios.
	\item La dirección considera que la inteligencia de negocios es otro centro de coste.
	\item El departamento de TI continúa preguntando a los usuarios finales sobre las necesidades de los informes.
	\item El sistema de BI esta soportado por help desk.
	\item No hay diferencia entre BI y gestión de rendimiento.
	\item No es posible medir el uso de inteligencia de negocio.
	\item Se considera que la estrategia para el datawarehouse es la misma que para que el sistema de inteligencia de negocio.

	\item No hay un plan para desarrollar, contratar, retener y aumentar el equipo de BI.
	\item No se conoce si la empresa tiene una estrategia para el BI.
	\item No existe un responsable funcional.
	\item No existe un centro de competencia.
	\item No hay un plan de formación real y consistente del uso de las herramientas.
	\item Los usuarios creen que la información del datawarehouse no es correcta.
	\end{itemize}


%%****
	\subsection{Ausencia de una estrategia de BI}
	Desarrollar una estrategia de negocio es un proceso a largo plazo que incluyen múltiples actividades donde podríamos destacar:
\\
\\
	\begin{itemize}
	\item Crear un centro de competencia de BI. Tiene el objetivo de armonizar conocimientos en tecnologías, metodologías, estrategias 
	con la presencia de un sponsor a nivel ejecutivo y analistas de negocio implicados y que tengan responsabilidades compartidas en 
	éxitos y fracasos.
	\item Establecer estándares de BI en la organización para racionalizar tanto las tecnologías existentes como las futuras adquisiciones.
	\item Desarrollar un Framework de métricas a nivel empresarial.
	\item Revisar y evaluar el portafolio actual de soluciones en un contexto de riesgo/recompensa.
	\item Aprender de los éxitos y fracasos de otras empresas revisando casos de estudio y consultar a las empresas del sector para 
	determinar que a funcionado y que no.
	\item Poner atención a las necesidades que requieren BI en la organización porque se acostumbra a satisfacer a los usuarios o 
	departamentos que gritan mas fuerte. Es decir, dar más atención a todas las áreas en atención en solución BI.
	
	\end{itemize}



%%-----------------------------------------------------------------------------
	\subsection{Analitica de Negocio (BA) }
	CONTENIDO
	\begin{itemize}
	\item 1. 
	\item 2. 
	\item 3. 
	\end{itemize}



%%-----------------------------------------------------------------------------
	

%%----------------------------------------------------------------------------------------------------------------------------------------------------------
%%FIN Marco Teórico




%CONCLUSIONES
\section{Conclusiones}
\subsection{Conclusión }	
%%****
Como conclusion acerca de Business Intelligence es que nos sirve para poderdar soluciones a ciertas igconitas a la organizacion que 
podria pasar por un momento donde se carece de una estrategia de BI.
Sobre el ¿Qué esta pasando?¿Qué pasa ahora?¿Porqué pasó?¿Que pasará? y asi mismo pueda darse una solución a la problematica 
existente dentro de sus áreas.





%RECOMENDACIONES
\section{Recomendaciones}	
CONTENIDO






	
	\newpage
	\bibliographystyle{apalike}
	\bibliography{BIBLIOGRAFIA}	
%\citep{referenciarobles2}  


\end{document}

=======

\documentclass[preprint,12pt]{elsarticle}

\usepackage[spanish]{babel}
\usepackage{amssymb}
\usepackage{graphicx}
\usepackage{lineno}
\usepackage[utf8]{inputenc}
\usepackage{url}
\usepackage{natbib}

\begin{document}
	
	\begin{frontmatter}

		\title{\huge  COMPARATIVA ENTRE INTELIGENCIA DE NEGOCIO (BI) Y ANALITICA DE NEGOCIO(BA) }
		\author{Robles Flores, Anthony Richard	                (2016056192)}
		\author{Estrella Palacios, Katherine Lizbeth			(2015050948)}
		\author{Sosa Bedoya, Sharon					(2016054460)}
		\author{Torres Beltran , Joihanna				(2015053235)}
		\address{Tacna, Perú}
		


%%INICIO abstract
\begin{abstract}
The context of the information society has led to the need for better, faster and more efficient methods 
to extract and transform the data of an organization into information and distribution along the value chain.
\\
In this article we will be able to detect the basic concepts about business intelligence (Business Intelligence) 
where it responds to it as a need and we can understand in a first approach that it is an evolution of decision 
support systems (DSS).
\end{abstract}
%%FIN abstract


\end{frontmatter}

%%INICIO Resumen
\section{Resumen}
El contexto de la sociedad de la información a propiciado tener la necesidad de mejores, más rápido y eficientes 
métodos para extraer y transformar los datos de una organización en información y distribuirla a lo largo de la 
cadena de valor.
\\
\\
En este articulo podremos apreciar los conceptos base acerca de la inteligencia de negocio (Business Intelligence)
 donde responde a esta como una necesidad y podemos entender en una primera aproximación que es una 
evolución de los sistemas de soporte a la decisiones (DSS).
%%FIN Resumen


%%INICIO Introducción
\section{Introducción}
CONTENIDO
%%FIN Introducción


%%INICIO Marco Teórico
\section{Marco Teórico}

%%----------------------------------------------------------------------------------------------------------------------------------------------------------
	\subsection{Inteligencia de Negocios (BI)}
	Hay que tomar en cuenta que este concepto Business Intelligence es un tema que viene desde octubre de 1958 
por Hans Peter Luhn (Investigador de IBM). Este concepto ha evolucionado aunando diferentes tecnologías, metodologías
 y términos.\cite{referenciarobles1}
\\
\\
Business Intelligence  es un conjunto de metodologías, aplicaciones, prácticas y capacidades enfocadas a la creación
 y administración de información que permite tomar las mejores decisiones a los usuarios en una organización.
\\
\\
Algunas de las tecnologías que forman parte de Business Intelligence son:
\\
\\
	\begin{itemize}
	\item Data wareHouse 
	\item Reporting 
	\item Análisis OLAP
	\item Análisis Visual
	\item Análisis Predictivo 
	\item Cuadro de mando
	\item Cuadro de mando integral
	\item Minera de datos 
	\item Gestión de rendimiento
	\item Reglas de negocio
	\item Dashboards 
	\item Integracion de Datos
	\end{itemize}



%%****
	\subsection{Beneficios de un sistema de Inteligencia de Negocio (BI)}
	La implantación de estos sistemas de información proporciona diversos beneficios entre los que podemos destacar:
\\
\\
	\begin{itemize}
	\item Crear un circulo virtuoso de la información (Donde los datos se transforman en información que permitirá 
		generar conocimiento para una toma de decisiones que se traducirán en mejores resultados y que 
		generarán nuevos datos).
	\item Permite una visión única, conformada, histórica, persistente y de calidad de toda la información. 
	\item Crear y manejar métricas, indicadores claves de rendimiento, e indicadores claves de meta fundamentales 
		para la empresa.
	\item Aportar información actualizada tanto a nivel agregado como en detalle.
	\item Reducir el diferencial de orientación de negocio entre el departamento de TI y la organización. 
	\item Mejorar comprensión y documentación de los sistemas de información en el contexto de una organización.
	\item Mejorar la competitividad de la organización como resultado de ser capaces de diferenciar lo relevante sobre 
		lo superfluo, acceder más rápido a la información, tener mayor agilidad en la toma de decisiones.
	\end{itemize}
%%****
	\subsection{La necesidad de la Inteligencia de Negocio}
	Existen situaciones en las que la implantación de un sistema Business Intelligence  resulta adecuada. Podríamos destacar 
	entre las que existen son:
\\
\\
	\begin{itemize}
	\item La toma de decisiones se realiza de forma intuitiva en la organización resultados y que generarán nuevos datos).
	\item Identificación de problema de calidad de información
	\item Uso de Excel como repositorios de información corporativos o de usuario o conocido como el Excel Caos.
	\item Necesidad de cruzar información de forma ágil entre departamentos.
	\item Evitar sirios de información.
	\item Las campañas de marketing no son efectivas por la información base usada. 
	\item Existe demasiada información en la organización para ser analizada de la forma habitual (se alcanzó la masa crítica de datos).
	\end{itemize}
%%****
	\subsection{Estrategia de Business Intelligence}
	Desplegar un proyecto de inteligencia de negocio en una organización no es un proceso sencillo. Las buenas practicas dicen 
	que para llegar a un buen puerto, es necesario tener una estrategia de inteligencia de negocio que coordine de forma efectiva 
	las tecnologías, el uso, los procesos de madurez.\cite{referenciarobles2}
%%****
	\subsection{Ausencia de una estrategia de BI}
	Es posible detectar que no existe estrategia definida a través de los siguientes puntos:
\\
\\
	\begin{itemize}
	\item Los usuarios identifican al departamento de TI como el origen de problemas de inteligencia de negocios.
	\item La dirección considera que la inteligencia de negocios es otro centro de coste.
	\item El departamento de TI continúa preguntando a los usuarios finales sobre las necesidades de los informes.
	\item El sistema de BI esta soportado por help desk.
	\item No hay diferencia entre BI y gestión de rendimiento.
	\item No es posible medir el uso de inteligencia de negocio.
	\item Se considera que la estrategia para el datawarehouse es la misma que para que el sistema de inteligencia de negocio.

	\item No hay un plan para desarrollar, contratar, retener y aumentar el equipo de BI.
	\item No se conoce si la empresa tiene una estrategia para el BI.
	\item No existe un responsable funcional.
	\item No existe un centro de competencia.
	\item No hay un plan de formación real y consistente del uso de las herramientas.
	\item Los usuarios creen que la información del datawarehouse no es correcta.
	\end{itemize}


%%****
	\subsection{Ausencia de una estrategia de BI}
	Desarrollar una estrategia de negocio es un proceso a largo plazo que incluyen múltiples actividades donde podríamos destacar:
\\
\\
	\begin{itemize}
	\item Crear un centro de competencia de BI. Tiene el objetivo de armonizar conocimientos en tecnologías, metodologías, estrategias 
	con la presencia de un sponsor a nivel ejecutivo y analistas de negocio implicados y que tengan responsabilidades compartidas en 
	éxitos y fracasos.
	\item Establecer estándares de BI en la organización para racionalizar tanto las tecnologías existentes como las futuras adquisiciones.
	\item Desarrollar un Framework de métricas a nivel empresarial.
	\item Revisar y evaluar el portafolio actual de soluciones en un contexto de riesgo/recompensa.
	\item Aprender de los éxitos y fracasos de otras empresas revisando casos de estudio y consultar a las empresas del sector para 
	determinar que a funcionado y que no.
	\item Poner atención a las necesidades que requieren BI en la organización porque se acostumbra a satisfacer a los usuarios o 
	departamentos que gritan mas fuerte. Es decir, dar más atención a todas las áreas en atención en solución BI.
	
	\end{itemize}



%%-----------------------------------------------------------------------------
	\subsection{Analitica de Negocio (BA) }
	CONTENIDO
	\begin{itemize}
	\item 1. 
	\item 2. 
	\item 3. 
	\end{itemize}



%%-----------------------------------------------------------------------------
	

%%----------------------------------------------------------------------------------------------------------------------------------------------------------
%%FIN Marco Teórico




%CONCLUSIONES
\section{Conclusiones}
\subsection{Conclusión }	
%%****
Como conclusion acerca de Business Intelligence es que nos sirve para poderdar soluciones a ciertas igconitas a la organizacion que 
podria pasar por un momento donde se carece de una estrategia de BI.
Sobre el ¿Qué esta pasando?¿Qué pasa ahora?¿Porqué pasó?¿Que pasará? y asi mismo pueda darse una solución a la problematica 
existente dentro de sus áreas.





%RECOMENDACIONES
\section{Recomendaciones}	
CONTENIDO






	
	\newpage
	\bibliographystyle{apalike}
	\bibliography{BIBLIOGRAFIA}	
%\citep{referenciarobles2}  


\end{document}

>>>>>>> 245e4fa5bd6bacc2861b58fecfc082917b5bc158
